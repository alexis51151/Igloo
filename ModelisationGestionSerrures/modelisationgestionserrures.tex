\documentclass[11pt,article]{article}
\usepackage[utf8]{inputenc}
\usepackage[T1]{fontenc} % caractères accentués en entrée, dans emacs
\usepackage[french]{babel}
\FrenchFootnotes
\selectlanguage{french}
\usepackage{a4wide} % possibilité d'utiliser toute la page a4
% selon GUT#33, avril 2007, page 13, empagement
% largeur des textes (ou justification) = 15cm
% hauteur du rectangle d'empagement = 23cm
% blanc de couture = 2/5 (21-15) = 2.4 = inner = right
% blanc de grand fond = 3/5 (21-15) = outer = left
% blanc de tête = 2/5 (29,7-23) = top
% blanc de pied = 3/5 (29,7-23) = bottom
%\usepackage[a4paper,twoside=true,right=2.4cm,left=3.6cm,top=2.68cm,bottom=4.02cm]{geometry} 
% selon CFSE 2006
% - largeur des textes (ou justification) : 16cm (2cm de marge, et 1cm
%   de reliure) ;
% - hauteur des textes, y compris les notes : 23cm (2,5cm de marge
%   haute et 2cm de marge basse) ; 1ère page de : 36pts
%   d'espacement avant le titre ;
\oddsidemargin   -4mm           % 3cm a gauche des impaires
\evensidemargin   4mm           % 2cm a gauche des paires
\topmargin       -18mm          % 2.5cm en haut
\headheight       13mm          % taille de l'entete (lignes)
\headsep          24pt          % espace entre entete et texte
\footskip         30pt          % espace entre pied de page et texte
\textheight      230mm          % longeur du texte
\textwidth       160mm          % largeur du texte
\parskip 1pt                    % pas de sauts entre paragraphes
%\parindent 0pt                  % largeur de l'indentation
\usepackage{graphicx} % figure postcript avec latex,
		      % figure png avec pdflatex, au lieu d'utiliser epsfig
\usepackage[usenames,dvipsnames,table]{xcolor}
\usepackage{paralist}
\usepackage{ifthen}
\usepackage{amssymb}
\usepackage{amsfonts}
\usepackage{amsmath}
\usepackage{eurosym}
\usepackage{textcomp}
\usepackage{listings}
\lstset{language=Java,numbers=left,numberstyle=\tiny,stepnumber=4,numbersep=5pt,xleftmargin=5pt}

\usepackage{alltt}
\usepackage{longtable}

% adjust word spacing less strictly
% as result, some spaces between words may be a bit too large,
% but long words will be placed properly.
\sloppy

\newcommand{\cmt}[1]{\texttt{<}\textbf{--~#1~--}\texttt{>}}

\usepackage{lineno}
\usepackage{xspace}

\setlength{\marginparwidth}{1cm}
\setlength{\marginparsep}{10pt}
\reversemarginpar
\newcounter{usecasehaute}
\newcommand{\haute}{Haute}
\newcommand{\moyenne}{Moyenne}
\newcommand{\basse}{basse}
\newcommand{\usecase}[4]{\item \marginpar{\vspace{5pt}\ifthenelse{\equal{#1}{Haute}}{\centering\textsc{#1}\stepcounter{usecasehaute}\newline n$^{\circ}$ \theusecasehaute}{\ifthenelse{\equal{#1}{Moyenne}}{#1}{\small #1}}} #2 \begin{itemize}\item précondition~: #3 \item postcondition~: #4\end{itemize}}
\newcommand{\priorityusecase}[2]{\item \marginpar{\vspace{5pt}\ifthenelse{\equal{#1}{Haute}}{\centering\textsc{#1}\stepcounter{usecasehaute}\newline n$^{\circ}$ \theusecasehaute}{\ifthenelse{\equal{#1}{Moyenne}}{#1}{\small #1}}} #2}
\newcommand{\casusecase}[4]{\usecase{#1}{#2}{#3}{#4}}

\newcommand{\nullvalue}{\textsf{null}\xspace}
\newcommand{\emptyvalue}{\ensuremath\mathrm{vide}}
\newcommand{\invariant}{\ensuremath\mathrm{invariant}}

\newcommand{\gestionclefshotel}{\textsc{GestionClefsHôtel}\xspace}
\newcommand{\gestionserrures}{\textsc{GestionSerrures}\xspace}

\begin{document}
\title{Système \gestionserrures pour la gestion des serrures programmables}
\author{Denis Conan}
\date{Janvier 2020}
\maketitle

\newpage

\tableofcontents

\newpage

Pour tester notre logiciel \gestionclefshotel, nous avons besoin de
simuler des serrures reprogrammables; c'est le rôle du système
\gestionserrures, que nous présentons maintenant. Le système
\gestionserrures est entièrement développé; le code est disponible
dans le projet fourni au départ (cf.~le
paquetage \begin{small}\textsf{eu.telecomsudparis.csc4102.gestionserrures}\end{small}).

\textbf{NB:} cette section n'est pas à lire dès le début du module,
mais au fur et à mesure de l'avancée dans le Sprint~1. Plus
précisément, nous détaillons:
\begin{compactitem}
\item pour la séance~2, la spécification avec les fonctionnalités de
simulation disponibles avec leurs préconditions et postconditions
(dans la section~\ref{SS_simserrures_specification}),
\item pour la séance~2, la préparation des tests de validation sous la
  forme de tables de décision (dans la
  section~\ref{SS_simserrures_preparationtestsvalidation}),
\item pour la séance~3, les éléments de conception permettant de
  comprendre et d'utiliser \gestionserrures (dans la
  section~\ref{SS_simserrures_conception}),
\item pour les séances~5 et~6, dans le code, la mise en œuvre du
système ainsi que de ses tests unitaires et de validation.
\end{compactitem}

\section{Spécification}
\label{SS_simserrures_specification}

Les fonctionnalités sont décrites dans un diagramme de cas
d'utilisation dans la
section~\ref{SS_simserrures_specification_diag_cas_utilisation}, et
leurs préconditions et postconditions dans la
section~\ref{SS_simserrures_specification_pre_post_conditions}.

\subsection{Diagramme de cas d'utilisation}
\label{SS_simserrures_specification_diag_cas_utilisation}

Le système \gestionserrures possède un seul acteur car c'est un
système de simulation, donc sans rôle particulier. L'utilisateur peut
créer, lister et supprimer les serrures. Il peut aussi ré-initialiser
la serrure en fournissant une nouvelle graine pour la génération des
clefs. Enfin, il peut simuler la présentation d'un badge pour essayer
d'ouvrir la porte. Le diagramme de cas d'utilisation est présenté dans
la figure qui suit.

\begin{figure}[!ht]
\begin{center}
\includegraphics[scale=0.5]{DiagrammesDeCasDUtilisation/gestionclefshotel_gestionserrures_uml_diag_cas_utilisation}
\caption{Diagramme de cas d'utilisation du système \gestionserrures}
\end{center}
\label{F_usecase_umlet_simserrures}
\end{figure}

\newpage

\subsection{Priorités, préconditions et postconditions des cas d'utilisation}
\label{SS_simserrures_specification_pre_post_conditions}

Nous donnons les préconditions et postconditions pour les cas
d'utilisation de priorité <<~\haute~>>. Pour les autres, nous
indiquons uniquement leur niveau de priorité.

\bigskip
\begin{compactitem}
\usecase{\haute}{Créer une serrure}
        %% précondition
        {identifiant de la serrure bien formé (non \nullvalue et non
          vide) $\land$ serrure avec cet identifiant inexistante}
        %% postcondition
        {serrure avec cet identifiant existante $\land$ les deux clefs
          sont les deux premières valeurs générées à partir de la
          graine et du sel}
        
\smallskip

\usecase{\haute}{Ré-initialiser les clefs d'une serrure}
        {identifiant de la serrure bien formé (non \nullvalue et non vide)
          $\land$ serrure avec cet identifiant existante $\land$ nouvelle
          graine de la serrure bien formée (non \nullvalue et non
          vide) $\land$ nouvelle graine ou nouveau sel différents}
        {une nouvelle graine $\lor$ un nouveau sel
          \newline$\land$ première clef de la nouvelle
          paire $\neq$ première clef de l'ancienne paire \newline$\land$
          première clef de la nouvelle paire $\neq$ seconde clef de
          l'ancienne paire \newline$\land$ seconde clef de la nouvelle
          paire $\neq$ première clef de l'ancienne paire \newline$\land$
          seconde clef de la nouvelle paire $\neq$ seconde clef de
          l'ancienne paire}

\usecase{\haute}{Ré-initialiser les clefs d'une serrure}
        %% précondition
        {identifiant de la serrure bien formé (non \nullvalue et non
          vide) $\land$ serrure avec cet identifiant existante $\land$
        graine de la serrure bien formé (non \nullvalue et non vide)}
        %% postcondition
        {serrure avec ce identifiant existante $\land$ les deux clefs
          sont les deux premières valeurs générées à partir de la graine}

\smallskip

\priorityusecase{\moyenne}{Obtenir la première clef d'une serrure}

\smallskip

\priorityusecase{\moyenne}{Obtenir la seconde clef d'une serrure}

\smallskip

\priorityusecase{\moyenne}{Lister les serrures}

\smallskip

\usecase{\haute}{Tester une serrure avec deux clefs}
        %% précondition
        {identifiant de la serrure bien formé (non \nullvalue et non
          vide) $\land$ serrure avec cet identifiant existante $\land$
          valeurs fournies pour les clefs $K_1$ et $K_2$ bien formées
          (non \nullvalue et non vides)}
        %% postcondition
        {~\newline
          $\Bigl[K_1$ et $K_2$ fournies identiques respectivement aux première
            et seconde clefs de la serrure $\lor$ $K_1$ fournie est
            identique à la seconde clef de la serrure$\Bigr]$}

\smallskip

\usecase{\haute}{Essayer d'ouvrir la porte avec un badge}
        %% précondition
        {identifiant de la serrure bien formé (non \nullvalue et non
          vide) $\land$ serrure avec cet identifiant existante $\land$
          valeur des clefs $K_1$ et $K_2$ bien formées (non \nullvalue
          et non vides) $\land$
          \newline
          $\Bigl[K_1$ et $K_2$ fournies identiques respectivement aux
            première et seconde clefs de la serrure  $\lor K_1$
            fournie était identique à la seconde clef de la serrure$\Bigr]$}
        %% postcondition
        {
          \newline
          $\bigl(K_1$ et $K_2$ fournies identiques respectivement aux
          première et seconde clefs de la serrure $\implies$ la porte
          est ouverte et les clefs de la serrure reste inchangées$\bigr)$
          \newline
          $\lor\bigl(K_1$ fournie était identique à la seconde clef de
          la serrure $\implies$ la première clef de la serrure
          devient $K_1$ et une seconde clef pour la serrure$\bigr)$
          \newline
          $\lor$ la porte ne s'ouvre pas}

\smallskip

\priorityusecase{\basse}{Supprimer une serrure}

\end{compactitem}

\section{Préparation des tests de validation}
\label{SS_simserrures_preparationtestsvalidation}

Pour les cas d'utilisation de priorité haute et les plus
<<~intéressants~>> car les plus sensibles ou risqués, nous préparons
des tests de validation.

\begin{table}[htbp!]
\begin{center}
\begin{tabular}{|p{0.6\linewidth}|c|c|c|c|}
\hline
Numéro de test
&1&2&3&4\\
\hline
\hline
Identifiant de la serrure bien formé (non \nullvalue, non vide)
&F&T&T&T\\
\hline
Graine de la serrure bien formé (non \nullvalue, non vide)
& &F&T&T\\
\hline
Serrure avec cet identifiant inexistante
& & &F&T\\
\hline
\hline
Création acceptée
&F&F&F&T\\
\hline
\hline
Nombre de jeux de test 
&2&2&1&1\\
\hline
\end{tabular}
\caption{Cas d'utilisation <<~créer une serrure~>>}
\end{center}
\end{table}

\begin{table}[htbp!]
\begin{tabular}{|p{0.6\linewidth}|c|c|c|c|c|}
\hline
Numéro de test
&1&2&3&4&5\\
\hline
\hline
Identifiant/code de la serrure bien formé (non \nullvalue et non vide)
&F&T&T&T&T\\
\hline
Serrure existante avec ce code
& &F&T&T&T\\
\hline
Nouvelle graine pour la génération des clefs de la chambre bien formée (non \nullvalue et non vide)
& & &F&T&T\\
\hline
Nouvelle graine ou nouveau sel différents
& & & &F&T\\
\hline
\hline
Génération de la nouvelle paire de clefs acceptée
&F&F&F&F&T\\
\hline
Nouvelle graine $\neq$ ancienne graine
\newline$\lor$ nouveau sel $\neq$ ancien sel
& & & & &T\\
\hline
Première clef de la nouvelle paire $\neq$ première clef de l'ancienne
paire \newline$\land$ première clef de la nouvelle paire $\neq$ seconde clef
de l'ancienne paire
& & & & &T\\
\hline
Seconde clef de la nouvelle paire $\neq$ première clef de l'ancienne
paire \newline$\land$ seconde clef de la nouvelle paire $\neq$ seconde clef de
l'ancienne paire
& & & & &T\\
\hline
\hline
Nombre de jeux de test 
&2&1&2&1&3\\
\hline
\end{tabular}
\caption{Cas d'utilisation <<~ré-initialiser les clefs d'une
  serrure~>>. Le jeu de tests du test~5 comprend trois tests:
  (1)~graine différente, (2)~sel différent, et (3)~graine et sel
  différents.}
\end{table}

\begin{table}[htbp!]
\begin{center}
\begin{tabular}{|p{0.8\linewidth}|c|c|c|c|c|}
\hline
Numéro de test
&1&2&3&4&5\\
\hline
\hline
Identifiant de la serrure bien formé (non \nullvalue, non vide)
&F&T&T&T&T\\
\hline
Serrure avec cet identifiant existante
& &F&T&T&T\\
\hline
Deux clefs bien formées (non \nullvalue, non vides)
& & &F&T&T\\
\hline
{\small$(K_1$ et $K_2$ fournies identiques respectivement aux
première et seconde clefs de la serrure$) \lor (K_1$ fournie est
identique à la seconde clef de la serrure$)$}
& & & &F&T\\
\hline
La porte peut s'ouvrire
&F&F&F&F&T\\
\hline
\hline
Nombre de jeux de test 
&2&1&4&5&3\\
\hline
\end{tabular}
\caption{Cas d'utilisation <<~tester une serrure avec deux clefs~>>}
\end{center}
\end{table}

Pour le calcul des jeux de test des tests~4 et~5 du cas d'utilisation
<<~tester une serrure avec deux clefs~>>, considérons tout d'abord
l'invariant suivant qui se déduit du texte du brevet: les deux clefs
de la serrure ne sont pas identiques. En considérant que l'invariant
est vérifié (c.-à-d., nous nous engageons dans la suite à vérifier par
assertion cet invariant dans le code), la table~3 présente toutes les
combinaisons possibles entre les clefs fournies ($K_1$ et $K_2$) et
celles de la serrure ($\mathit{KS}_1$ et $\mathit{KS}_2$).

\begin{table}[htbp!]
\begin{center}
\begin{tabular}{l||c|c|c|c||p{0.4\textwidth}}
\hline
\textbf{\#} & \multicolumn{4}{c||}{\textbf{Conditions}} & \textbf{Résultat ou\newline raison d'impossibilité du test}\\
\hline
\hline
%% 1
&
$K_1=\mathit{KS}_1$ & $K_2=\mathit{KS}_2$ & $K_1=\mathit{KS}_2$ &
$K_2=\mathit{KS}_1$ & Impossible car non-respect de l'invariant de la serrure\\
\hline
%% 2
&
" & " & " & $K_2\neq\mathit{KS}_1$ & Impossible car contradiction\newline ($K_2=\mathit{KS}_2=K_1=\mathit{KS}_1\neq{}K_2$)\\
\hline
%% 3
&
" & " & $K_1\neq\mathit{KS}_2$ & $K_2=\mathit{KS}_1$ & Impossible car
contradiction\newline ($K_1=\mathit{KS}_1=K_2=\mathit{KS}_2\neq{K_1}$)\\
\hline
%% 4
&
" & " & " & $K_2\neq\mathit{KS}_1$ & Impossible car non-respect de l'invariant de la serrure\\
\hline
%% 5
&
" & $K_2\neq\mathit{KS}_2$ & $K_1=\mathit{KS}_2$ & $K_2=\mathit{KS}_1$
& Impossible car contradiction\newline ($K_2=\mathit{KS}_1=K_1=\mathit{KS}_2\neq{}K_2$)\\
\hline
%% 6
&
" & " & " & $K_2\neq\mathit{KS}_1$ & Impossible car non-respect de l'invariant de la serrure\\
\hline
%% 7
F1&
" & " & $K_1\neq\mathit{KS}_2$ & $K_2=\mathit{KS}_1$ & $\mathit{false}$\\
\hline
%% 8
F2&
" & " & " & $K_2\neq\mathit{KS}_1$ & $\mathit{false}$\\
\hline
%% 9
&
$K_1\neq\mathit{KS}_1$ & $K_2=\mathit{KS}_2$ & $K_1=\mathit{KS}_2$ &
$K_2=\mathit{KS}_1$ & Impossible car contradiction\newline ($K_1=\mathit{KS}_2=K_2=\mathit{KS}_1\neq{}K_1$)\\
\hline
%% 10
T1&
" & " & " & $K_2\neq\mathit{KS}_1$ & $\mathit{true}$\\
\hline
%% 11
&
" & " & $K_1\neq\mathit{KS}_2$ & $K_2=\mathit{KS}_1$ & Impossible car non-respect de l'invariant de la serrure\\
\hline
%% 12
F3&
" & " & " & $K_2\neq\mathit{KS}_1$ & $\mathit{false}$\\
\hline
%% 13
T2&
" & $K_2\neq\mathit{KS}_2$ & $K_1=\mathit{KS}_2$ & $K_2=\mathit{KS}_1$
& $\mathit{true}$\\
\hline
%% 14
T3&
" & " & " & $K_2\neq\mathit{KS}_1$ & $\mathit{true}$\\
\hline
%% 15
F4&
" & " & $K_1\neq\mathit{KS}_2$ & $K_2=\mathit{KS}_1$ & $\mathit{false}$\\
\hline
%% 16
F5&
" & " & " & $K_2\neq\mathit{KS}_1$ & $\mathit{false}$\\
\hline
\end{tabular}
\end{center}
\caption{Jeux de test pour l'exhaustivité des test des cas
d'utilisation <<~tester une serrure avec deux clefs~>> et <<~essayer
d'ouvrir la porte avec un badge~>> (<<~FX~>> et <<~TX~>> servent à
numéroter les tests; <<~"~>> signifie <<~\textit{idem}~>>)}
\end{table}

\begin{table}[htbp!]
\begin{center}
\begin{tabular}{|p{0.8\linewidth}|c|c|c|c|c|}
\hline
Numéro de test
&1&2&3&4&5\\
\hline
\hline
Identifiant de la serrure bien formé (non \nullvalue, non vide)
&F&T&T&T&T\\
\hline
Serrure avec cet identifiant existante
& &F&T&T&T\\
\hline
Deux clefs bien formées (non \nullvalue, non vides)
& & &F&T&T\\
\hline
{\small$(K_1$ et $K_2$ fournies identiques respectivement aux
première et seconde clefs de la serrure$) \lor (K_1$ fournie est
identique à la seconde clef de la serrure$)$}
& & & &F&T\\
\hline
{\small
$\bigl(K_1$ et $K_2$ fournies identiques respectivement aux
première et seconde clefs de la serrure $\implies$ la porte est
ouverte et les clefs de la serrure reste inchangées$\bigr)$
\newline
$\lor\bigl(K_1$ fournie était identique à la
seconde clef de la serrure $\implies$ la première clef de la serrure
devient $K_1$ et une seconde clef est générée pour la serrure$\bigr)$
\newline
$\lor$ la porte ne s'ouvre pas}
&F&F&F&F&T\\
\hline
\hline
Nombre de jeux de test
&2&1&2&5&3\\
\hline
\end{tabular}
\caption{Cas d'utilisation <<~essayer d'ouvrir la porte avec un badge~>>}
\end{center}
\end{table}

\newpage~\newpage

\section{Conception}
\label{SS_simserrures_conception}

\subsection{Diagrammes de séquence}

Le diagramme de classes de la figure qui suit contient la façade du
système: la classe \textsf{GestionSerrures}, ainsi que la classe
\textsf{Serrures} qui réalise le concept de serrure programmable avec
deux clefs.

\begin{figure}[!ht]
\begin{center}
\includegraphics[scale=0.5]{DiagrammesDeClasses/gestionclefshotel_gestionserrures_uml_diag_classes}
\caption{Diagramme de classes}
\end{center}
\label{umlet_simserrures_diag_classes}
\end{figure}

\newpage

\subsection{Diagrammes de séquence}

Voici la description textuelle du cas d'utilisation <<~créer une serrure~>>:
\begin{compactitem}
\item arguments en entrée: identifiant de la serrure, la graine de la
  serrure pour la génération des clefs, le sel de la serrure pour la
  génération des clefs
\item rappel de la précondition: identifiant de la serrure bien
  formé (non \nullvalue et non vide) $\land$ graine bien formée (non
  \nullvalue et non vide) $\land$ serrure avec cet identifiant
  inexistante
\item algorithme:
\begin{compactenum}
\item vérifier les arguments
\item vérifier que la serrure est inexistante
\item instancier la serrure
  \begin{compactitem}
  \item générer deux nouvelles clefs
  \end{compactitem}
\item ajouter la serrure dans la collection des serrures
\end{compactenum}
\end{compactitem}

\begin{figure}[ht!]
\begin{center}
\includegraphics[scale=0.5]{DiagrammesDeSequence/gestionclefshotel_gestionserrures_uml_diag_sequence_creer_serrure}
\caption{Diagramme de séquence du cas d'utilisation <<~créer une serrure~>>}
\end{center}
\label{umlet_diag_sequence_creer_serrure}
\end{figure}

\newpage

Voici la description textuelle du cas d'utilisation <<~ré-initialiser une serrure~>>:
\begin{compactitem}
\item arguments en entrée: identifiant de la serrure, nouvelle graine,
  nouveau sel
\item rappel de la précondition: identifiant de la serrure bien formé
  (non \nullvalue et non vide) $\land$ nouvelle graine bien formée
  (non \nullvalue et non vide) $\land$ serrure avec cet identifiant
  existante
\item algorithme:
\begin{compactenum}
\item vérifier les arguments
\item vérifier que la serrure existe
\item ré-initialiser la serrure
  \begin{compactitem}
  \item générer deux nouvelles clefs
  \end{compactitem}
\end{compactenum}
\end{compactitem}

\begin{figure}[ht!]
\begin{center}
\includegraphics[scale=0.5]{DiagrammesDeSequence/gestionclefshotel_gestionserrures_uml_diag_sequence_reinitialiser_serrure}
\caption{Diagramme de séquence du cas d'utilisation <<~ré-initialiser une serrure~>>}
\end{center}
\label{umlet_diag_sequence_reinitialiser_serrure}
\end{figure}

\newpage

Voici la description textuelle du cas d'utilisation <<~tester une
serrure avec deux clefs~>>:
\begin{compactitem}
\item arguments en entrée: identifiant de la serrure, première clef,
  seconde clef
\item rappel de la précondition: identifiant de la serrure bien formé
  (non \nullvalue et non vide) $\land$ serrure avec cet identifiant
  existante $\land$ valeurs fournies pour les clefs $K_1$ et $K_2$
  bien formées (non \nullvalue et non vides)
\item algorithme:
\begin{compactenum}
\item vérifier les arguments
\item vérifier que la serrure existe
\item tester la serrure
\end{compactenum}
\end{compactitem}

\begin{figure}[ht!]
\begin{center}
\includegraphics[scale=0.5]{DiagrammesDeSequence/gestionclefshotel_gestionserrures_uml_diag_sequence_tester_serrure}
\caption{Diagramme de séquence du cas d'utilisation <<~tester une
  serrure avec deux clefs~>>}
\end{center}
\label{umlet_diag_sequence_tester_serrure}
\end{figure}

\newpage

Voici la description textuelle du cas d'utilisation <<~essayer
d'ouvrir une serrure avec deux clefs~>>:
\begin{compactitem}
\item arguments en entrée: identifiant de la serrure, première clef,
  seconde clef
\item rappel de la précondition: identifiant de la serrure bien formé
  (non \nullvalue et non vide) $\land$ serrure avec cet identifiant
  existante $\land$ valeurs fournies pour les clefs $K_1$ et $K_2$
  bien formées (non \nullvalue et non vides)
\item algorithme:
\begin{compactenum}
\item vérifier les arguments
\item vérifier que la serrure existe
\item essayer d'ouvrir la serrure
  \begin{compactitem}
  \item générer une nouvelle clef
  \end{compactitem}
\end{compactenum}
\end{compactitem}

\begin{figure}[ht!]
\begin{center}
\includegraphics[scale=0.5]{DiagrammesDeSequence/gestionclefshotel_gestionserrures_uml_diag_sequence_essayer_ouvrir_serrure}
\caption{Diagramme de séquence du cas d'utilisation <<~essayer d'ouvrir une
  serrure avec deux clefs~>>}
\end{center}
\label{umlet_diag_sequence_essayer_ouvrir_serrure}
\end{figure}

\newpage

\subsection{Fiches des classes}

\subsubsection{Classe \textsf{\gestionserrures}}

\begin{center}
\begin{longtable}{|p{15cm}|} 
\hline
\multicolumn{1}{|c|}{{\Large \textsf{\gestionserrures}}} \\
\hline
%\cmt{attributs}\\
$+$ \underline{MESSAGE\_PORTE\_OUVERT} : String = ``la porte est ouverte'' \\
$+$ \underline{MESSAGE\_PORTE\_FERMÉE} : String = ``la porte reste fermée'' \\
\cmt{attributs <<~association~>>}\\
$-$ serrures : collection de @Serrure \\
\hline
\cmt{constructeur} \\
$+$ \gestionserrures()\\
%$+$ destructeur()\\
$+$ invariant():booléen\\
\cmt{operations <<~cas d'utilisation~>>} \\
$+$ créerSerrure(String identifiant, String graine, int sel) \\
$+$ essayerDOuvrirUnePorte(String identifiant, byte[] premièreClef, byte[] secondeClef): String \\
$+$ invariant(): booléen \\
$+$ listeSerrures(): String[] \\
$+$ obtenirPremiereClefSerrure(String identifiant): byte[] \\
$+$ obtenirSecondeClefSerrure(String identifiant): byte[] \\
$+$ tester(String identifiant, byte[] premièreClef, byte[] secondeClef):booléen \\
%\cmt{opérations de recherche} \\
$-$ chercherSerrure(String identifiant): Serrure \\
\hline  
\end{longtable}%)
\end{center}

\subsubsection{Classe \textsf{Serrure}}

\begin{center}
\begin{longtable}{|p{15cm}|} 
\hline
\multicolumn{1}{|c|}{{\Large \textsf{Serrure}}} \\
\hline
%\cmt{attributs}\\
$-$ identifiant: String \\
$-$ graine: String \\
$-$ sel: int \\
$-$ premiereClef: byte[] \\
$-$ secondeClef: byte[] \\
%\cmt{attributs <<~association~>>}\\
\hline
\cmt{constructeur} \\
$+$ Serrure(String identifiant, String graine, int sel)\\
%$+$ destructeur()\\
$+$ invariant():booléen\\
$+$ essayerDOuvrirUnePorte(byte[] premièreClef, byte[] secondeClef): String \\
$+$ invariant(): booléen \\
$+$ getPremiereClef(): byte[] \\
$+$ getSecondeClef(): byte[] \\
$+$ tester(byte[] premièreClef, byte[] secondeClef):booléen \\
\hline  
\end{longtable}%)
\end{center}

\newpage

\section{Préparation des tests unitaires}

\subsection{Classe \textsf{Serrure}}

\newcommand{\identifiant}{\ensuremath\mathrm{identifiant}}
\newcommand{\graine}{\ensuremath\mathrm{graine}}
\newcommand{\sel}{\ensuremath\mathrm{sel}}
\newcommand{\premiereClef}{\ensuremath\mathrm{premiereClef}}
\newcommand{\secondeClef}{\ensuremath\mathrm{secondeClef}}
\begin{table}[!ht]
\begin{center}
\begin{tabular}{|p{0.4\linewidth}|c|c|c|}
\hline
Numéro de test
&1&2&3\\
\hline
\hline
\texttt{identifiant} $\neq \nullvalue \land \neg \emptyvalue$
&F&T&T\\
\hline
\texttt{graine} $\neq \nullvalue \land \neg \emptyvalue$
& &F&T\\
\hline
\hline
$\identifiant' = \identifiant$
& & &T\\
\hline
$\graine' = \graine$
& & &T\\
\hline
$\sel' = \sel$
& & &T\\
\hline
$\premiereClef' \neq \nullvalue \neg \emptyvalue$
& & &T\\
\hline
$\secondeClef' \neq \nullvalue \neg \emptyvalue$
& & &T\\
\hline
$\premiereClef' \neq \secondeClef$
& & &T\\
\hline
$\invariant$
& & &T\\
\hline
Levée d'une exception&\textsc{oui}&\textsc{oui}&\textsc{non}\\
\hline
\hline
Objet créé
&F&F&T\\
\hline
\hline
Nombre de jeux de test 
&2&2&1\\
\hline
\end{tabular}
\caption{Méthode \texttt{constructeurSerrure} de la
classe \texttt{Serrure}~>>}
\end{center}
\end{table}

\begin{table}[htbp!]
\begin{center}
\begin{tabular}{|p{0.8\linewidth}|c|c|c|}
\hline
Numéro de test
&1&2&3\\
\hline
\hline
Deux clefs bien formées (non \nullvalue, non vides)
&F&T&T\\
\hline
{\small$(K_1$ et $K_2$ fournies identiques respectivement aux
première et seconde clefs de la serrure$) \lor (K_1$ fournie est
identique à la seconde clef de la serrure$)$}
& &F&T\\
\hline
\hline
$\premiereClef' = \premiereClef$
& & &T\\
\hline
$\secondeClef' = \secondeClef$
& & &T\\
\hline
$\invariant$
& & &T\\
\hline
Levée d'une exception&\textsc{oui}&\textsc{oui}&\textsc{non}\\
\hline
\hline
La porte peut s'ouvrire
&F&F&T\\
\hline
\hline
Nombre de jeux de test 
&4&5&3\\
\hline
\end{tabular}
\caption{Méthode \texttt{tester} de la classe
  \texttt{Serrure}~>>. Nous utilisons la même table pour calculer le
  nombre de tests dans les jeux de test des tests~2 et~3:
  5~combinaisons possibles pour \textsf{false} et~3 combinaisons
  possibles pour \textsf{true}.}
\end{center}
\end{table}

\begin{table}[htbp!]
\begin{center}
\begin{tabular}{|p{0.8\linewidth}|c|c|c|}
\hline
Numéro de test
&1&2&3\\
\hline
\hline
Deux clefs bien formées (non \nullvalue, non vides)
&F&T&T\\
\hline
{\small$(K_1$ et $K_2$ fournies identiques respectivement aux
première et seconde clefs de la serrure$) \lor (K_1$ fournie est
identique à la seconde clef de la serrure$)$}
& &F&T\\
\hline
{\small
$\bigl(K_1$ et $K_2$ fournies identiques respectivement aux
première et seconde clefs de la serrure $\implies$ la porte est
ouverte et les clefs de la serrure reste inchangées$\bigr)$
\newline
$\lor\bigl(K_1$ fournie était identique à la
seconde clef de la serrure $\implies$ la première clef de la serrure
devient $K_1$ et une seconde clef est générée pour la serrure$\bigr)$
\newline
$\lor$ la porte ne s'ouvre pas}
&F&F&T\\
\hline
\hline
Nombre de jeux de test
&2&5&3\\
\hline
\end{tabular}
\caption{Méthode \texttt{essayerDOuvrirLaPorte} de la classe
  \texttt{Serrure}~>>. Nous utilisons la même table pour calculer le
  nombre de tests dans les jeux de test des tests~2 et~3:
  5~combinaisons possibles pour \textsf{false} et~3 combinaisons
  possibles pour \textsf{true}.}
\end{center}
\end{table}

\begin{table}[!ht]
\begin{center}
\begin{tabular}{|p{0.4\linewidth}|c|c|c|}
\hline
Numéro de test
&1&2&3\\
\hline
\hline
\texttt{graine} $\neq \nullvalue \land \neg \emptyvalue$
&F&T&T\\
\hline
\texttt{graine} ou \texttt{sel} différents
& &F&T\\
\hline
\hline
$\identifiant' = \identifiant$
& &F&T\\
\hline
$\graine' \neq \graine \lor \sel' \neq \sel$
& & &T\\
\hline
$\premiereClef' \neq \premiereClef$
& & &T\\
\hline
$\premiereClef' \neq \secondeClef$
& & &T\\
\hline
$\secondeClef' \neq \premiereClef$
& & &T\\
\hline
$\secondeClef' \neq \secondeClef$
& & &T\\
\hline
$\premiereClef' \neq \secondeClef'$
& & &T\\
\hline
$\invariant$
& & &T\\
\hline
Levée d'une exception&\textsc{oui}&\textsc{oui}&\textsc{non}\\
\hline
\hline
Clefs ré-initialisées
&F&F&T\\
\hline
\hline
Nombre de jeux de test 
&2&1&3\\
\hline
\end{tabular}
\caption{Méthode \texttt{réInitialiser} de la classe
  \texttt{Serrure}~>>. Le test~3 comporte un jeu de trois tests selon
  que la graine ou le sel sont différents.}
\end{center}
\end{table}

\end{document}
